\addcontentsline{toc}{section}{Abstract}
\section*{Abstract}
Inertial Measurement Units (IMUs) are widely used in a
variety of applications such as Body Sensor Networks (BSNs) for orientation
estimation, however the gyroscope suffers from drift due to sensor bias and noise that 
when integrated accumulate over time. This \fixme{project} investigates a deep 
learning-based approach which aims to mitigate gyroscopic errors which can be 
integrated with sensor fusion techniques to achieve more accurate orientation 
estimates. The proposed deep-learning architecture leverages both neural 
networks and a temporal history to learn complex and nonlinear error patterns
in IMU data, exploring if it outperforms a standard Kalman Filter without learned 
corrections. The network outputs a correction for the incoming gyroscope sample and 
ad the measurement noise covariance dependent on the incoming acceleration and 
magnetometer updates. The data used in training, testing and validating the model 
come from simulations through MATLAB's Navigation Toolbox and from public datasets 
such as Berlin Robust Orientation Estimation Assessment Dataset (BROAD).