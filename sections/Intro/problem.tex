\subsection{Problem Statement: IMU Drift and Its Impact}
IMUs have shown promise in determining the orientation of an object in motion. However, IMUs 
suffer from a limitation called drift. IMU drift is characterised by the accumulation of errors
through the integration of the angular rate. The sources of these
errors include constant bias, scale factor errors and others expanded in \fixme{section}. 
Errors are  not exclusive to the gyroscope but also affect the accelerometer and magnetometer.
Drift is also dependent on the type of IMU that is used.
Lower cost/grade IMUs suffer from drift at a higher magnitude which results orientation 
inaccuracies much quicker compared to higher cost/grade IMUs.  
\todo{find cites for IMUs and add some metrics}
Errors then lead to inaccuracies in the orientation estimation of an object determined by the 
IMU. 

Kianifar et al. explored using IMUs  for automated orientation estimation in a clinical 
setting. They found that for rotation angles parallel to gravity, drift due to gyroscope bias
cannot be compensated by the accelerometer.\fixme{cite}. Even with multiple sensors, it is still challenging to find an accurate
orientation estimation. Thus it is important to try and address the orientation problem by addressing 
gyroscopic drift. 

Therefore, this project aims to address gyroscopic drift by using deep-learning methods to learn
complex and non-linear nature of biases and errors. 

\words{211}