\subsection{Gyroscope: Operations and Errors}
The gyroscope is the main component that is used to determine the orientation 
of the object. It measures the angular rate in its orthorgonal axes $\omega^x,\omega^y,\omega^z$
which is used to determined the object's orientation at a discrete time. The orientation is 
determined through the integration of angular rate. There are multiple different
ways to represent the orientation of the object, these being, Euler angles, rotation matrices, and
quaternions. Euler angles give rise to the singularity problem due to the loss
of one degree-of-freedom whenever two axes of the rotations are parallel \fixme{cite}.
Rotation matrices are also used, however they are less concise as they are
represented as 3x3 matrix, where 6 of these elements are redundant\fixme{cite}.
Therefore in this chapter, the quaternion representation is selected due to 
being more computationally efficient than the others \fixme{cite prof paper}.


\subsubsection{Angular Rate Integration}
Starting in continuous-time kinematics, we can define a quaternion that is represented by the 
angular rate, where $\omega$ is the real angular rate, $\omega_q(t)$ is the pure quaternion. 

\begin{equation}
\omega_q(t) = [0,\ \omega_x(t),\ \omega_y(t),\ \omega_z(t)]
\end{equation}

The orientation evolves according to\fixme{cite}

\begin{equation}
\dot{q}(t) = \frac{1}{2}q(t) \otimes \omega_q(t)
\end{equation}

where $\otimes$ denotes quaternion multiplication. The solution over $[t_0,t]$ can be written 
using the quaternion exponential, this assumes that the $\omega_q$ is constant over
$\tau$.

\begin{equation}
{q}(t) = q(t_0) \otimes \exp(\frac{1}{2}\int_{t_0}^{t} \omega_q(\tau) \,d\tau)
\end{equation}

These equations show that the orientation update is determined by integrating the angular rate
over time and mapping the resulting rotation into a quaternion via the exponential.

Moving to discrete-time kinematics, we can restate of defintions in terms
$\Delta t$. Here the orientation evolves according to


\begin{equation}
\hat{q}_k = \hat{q}_{k-1} \otimes \Delta q_k
\end{equation}

where $\Delta q_k$ is 

\begin{equation}
\Delta q_k = \exp(\frac{1}{2}\omega_{q,k} \Delta t) 
\end{equation}

This was all done by using an ideal $\omega_k$, if were to model the angular
rate from a gyroscope as \fixme{cite}

\begin{equation}
\omega_{m,k} = \omega_k + b_k + n_k 
\end{equation}

Where $\omega_{m,k}$ is the measured angular rate, $\omega_k$ is the true angular rate,
$b_k$ is the bias term, and $n_k$ is the noise term. The real quaternion orientation
update is as follows.

\begin{equation}
\hat{q}_k = \hat{q}_{k-1} \otimes \exp(\frac{1}{2}(\omega_k + b_k + n_k)_q \Delta t)
\end{equation}

This shows that the inclusion of the bias and noise accumulates over samples
as we move forward to the next k, $q_{k-1}$ will incorporate the previous error
terms. This results in an accumulated orientation error.






\subsubsection{Gyroscope Error Sources}
\label{sec:gyroerrors}
\subsubsection*{Constant Bias}
The bias of a gyroscope is the average output from the gyroscope when it is not undergoing any
rotation \fixme{cite}. This is measured in $ \degree /h$ and can be estimated by taking an average
of the output. As this bias is constant, the drift it causes grows linearly with time. 
However, as discussed further in this section other error sources can make this
difficult to determine. \fixme{Figure} shows how constant bias changes the output.

\begin{figure}[htp]
    \centering
    \includegraphics[width=5cm]{./figures/IMU/Howgood_bias.png}
    \caption{Bias Effect on Gyro \fixme{cite}}
    \label{fig:GyroBias}
\end{figure}


\subsubsection*{White Noise / Angle Random Walk (ARW)}
The gyroscope is also affected by some white noise \fixme{cite}. This white noise sequence is zero-mean uncorrelated
random variables between samples and across axes. When the gyroscope signal is integrated to obtain
an angle, this white noise produces an ARW.
The units of ARW is denoted by $ \degree / \sqrt{h}$, this shows that the deviation of angle error grows proportionally to $\sqrt{t}$.
As ARW is due to random variables, it is classified as the $1\sigma$ of the orientation error. Analog devices
shows this in the \fixme{figure}, where the ARW is $ 0.17\degree / \sqrt{h}$.

\begin{figure}[htp]
    \centering
    \includegraphics[width=5cm]{./figures/IMU/AD_ARW.png}
    \caption{Bias Effect on Gyro \fixme{cite}}
    \label{fig:GyroARW}
\end{figure}

\subsubsection*{Flicker Noise / Bias Stability}
The gyroscope suffers from flicker noise in electronics \fixme{cite}
Flicker noise is $1/f$, due to this the effects are observed at lower 
frequenices while, higher frequenices the noise is dominated by white noise.

\subsubsection*{Temperature}
Temperature fluctuations due to changes in the environment and sensor heating
induce a movement in the bias, this relationship is also highly non-linear \fixme{cite}. Therefore, it can be 
very difficult to model and subsequently subtract from the gyroscope measurements
compared to a constant bias.



 