\subsection{Gyroscope: Operations and Errors}
The gyroscope is the main component that is used to determine the orientation 
of the object. It measures the angular rate in its orthorgonal axes $\omega^x,\omega^y,\omega^z$
which is used to determined the object's orientation at a discrete time. The orientation is 
determined through the integration of angular rate.

\subsubsection{Angular Rate Integration}
Starting in continuous-time kinematics, we can define a quaternion that is represented by the 
angular rate, where $\omega_q(t)$ is the real angular rate. 

\begin{equation}
\mathbf{\omega_q(t)} = [0,\ \omega_x(t),\ \omega_y(t),\ \omega_z(t)]
\end{equation}

The orientation evolves according to

\begin{equation}
\dot{q}(t) = \frac{1}{2}q(t) \otimes \omega_q(t)
\end{equation}

where $\otimes$ denotes quaternion multiplication. The solution over $[t_0,t]$ can be written 
using the quaternion exponential

\begin{equation}
{q}(t) = q(t_0) \otimes \exp(\frac{1}{2}\int_{t_0}^{t} \omega_q(\tau) \,d\tau)
\end{equation}

These equations show that the orientation update is determined by integrating the angular rate
over time and mapping the resulting rotation into a quaternion via the exponential.

\subsubsection{Gyroscope Error Sources}
\subsubsection*{Constant Bias}
The bias of a gyroscope is the average output from the gyroscope when it is not undergoing any
rotation \fixme{cite}. This is measured in $ \degree /h$ and can be estimated by taking an average
of the output. However, as discussed further in this section other error sources can make this
difficult to determine. 

\subsubsection*{White Noise / Angle Random Walk (ARW)}
The gyroscope is also affected by some white noise that fluctuates at a higher rate than the
sampling rate of the sensor \fixme{cite}. This white noise sequence is zero-mean uncorrelated
random variables between samples and across axes. When the gyroscope signal is integrated to obtain
an angle, this white noise produces an ARW.
The units of ARW is denoted by $ \degree / \sqrt{h}$, where if this value is $0.2\degree / 
\sqrt{h}$, the ARW in one hour is $0.2 \degree / \sqrt{h}$ and in two hours is  $0.28\degree$
This shows that the deviation of angle error grows proportionally to $\sqrt{t}$.

\subsubsection*{Flicker Noise / Bias Stability}

\subsubsection*{Temperature}

\subsubsection*{Scale Factor}