\subsection{Accelerometer: Operations and Errors}
The accelerometer is another component used in determining an orientation estimate of an object.
It measures the specific force along its orthogonal axes $a^x,a^y,a^z$. These measurements can 
be rotated in the navigation frame, gravity vector is compensated, and through integration of the accelerometer,  a velocity and subsequently the position can be derived for
an object.
However, the focus of this project is to achieve drift-free orientation, and hence
this will not be covered. 

The main purpose of the accelerometer is to offer a gravity reference to the system such that,
an accelerometer is capable of providing drift-free inclination estimates \fixme{cite}.
However, the accelerometer, as stated above, does not only measure a gravity reference but also any 
linear accelerations that the object also experiences. Under conditions of high 
accelerations, the reference is unreliable and is only reliable for static or slowly moving 
objects \fixme{cite}. Additionally, the accelerometer cannot detect rotations about the gravity
vector, so it cannot provide a yaw/heading of the object.
However, like the gyroscope, it also suffers from a variety of errors which causes instability in the
output.

\subsubsection{Accelerometer Error Effects}
Accelerometers suffer from a very similar sources of errors to the gyroscope (\hyperref[sec:gyroerrors]{section 3.1.2}). Therefore, in this
section we will discuss the implications of these errors for our orientation estimates.

\subsubsection*{Constant Bias}
The effect of constant bias is that it causes an offset in the roll/pitch of the system.
Unlike constant bias in the gyroscope, it does not grow linearly with time but is a constant
offset irrespective of time elapsed.

\subsubsection*{High Linear Acceleration}
As discussed above, the largest problem is high linear accelerations which corrupts the gravity
reference. This is a dominant issue in high dynamic motions as the accelerometer will not be able
to correct the orientation estimation through fusion \fixme{section}, especially if gyroscopic drift is not addressed. 

A potential plan to address this will be discussed in \fixme{section}
