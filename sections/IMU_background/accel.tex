\subsection{Accelerometer: Operations and Errors}
The accelerometer is another component used in determining an orientation estimate of an object.
It measures the specific force along its orthogonal axes $a^x,a^y,a^z$. These measurements can 
be rotated in the navigation frame, gravity vector is subtracted, and through integration of the accelerometer,  a velocity and subsequently the position can be derived for
an object.
However, the focus of this project is to achieve drift-free orientation, and hence
this will not be covered. 

The main purpose of the accelerometer is to offer a gravity reference to the system such that,
an accelerometer is capable of providing drift-free inclination estimates \fixme{cite}.
However, the accelerometer, as stated above, does not only measure a gravity reference but also any 
linear and centripetal accelerations that the object also experiences. Under conditions of high 
accelerations, the reference is unreliable and is only reliable for static or slowly moving 
objects \fixme{cite}. Additionally, the accelerometer cannot detect rotations about the gravity
vector, so it cannot provide a yaw/heading of the object.
However, like the gyroscope, it also suffers from a variety of errors which causes drift in the
output.




\subsubsection{Accelerometer Error Sources}